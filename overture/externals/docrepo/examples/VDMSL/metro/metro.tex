\documentclass{article}
\usepackage{overture}%vdmsl-2e}

\title{Abstract Metro Specifications in VDM-SL}
\author{Sten Agerholm\\{\small IFAD, Forskerparken 10, DK-5230 Odense M,
  Denmark}\\{\small Email: sten@ifad.dk}}
\begin{document}
\maketitle

\section{Introduction}
Below I'll provide three different abstract specifications of a metro
door management system in VDM-SL. The purpose of the presentation is
to describe alternatives to the Metro specification developed in the
SPECTRUM project (Esprit project no.\ 23173) and presented in the
Proceedings of FMSP'98, published by ACM SIGSOFT, March
1998\footnote{Also available at {\small
    ftp://ftp.ifad.dk/pub/papers/final.ps.gz}.}. The present document
is a draft and the specifications have not tested, so there may be
bugs!

Basically, the metro door management system can be described as
follows. There are three actors: the train driver, the train itself,
and the doors. The train driver has four button that he can use to
control the train and the doors: accelerate, break, open and
close. Our main concern is the safety requirement that the train
should not be able to move while the doors are open.

\section{A First Straightforward Specification}
The specification below models actions (or buttons) as VDM-SL
operations. The state of the doors and the train is held in a VDM-SL state.

\documentclass{article}
\usepackage{overture}%vdmsl-2e}

\title{Abstract Metro Specifications in VDM-SL}
\author{Sten Agerholm\\{\small IFAD, Forskerparken 10, DK-5230 Odense M,
  Denmark}\\{\small Email: sten@ifad.dk}}
\begin{document}
\maketitle

\section{Introduction}
Below I'll provide three different abstract specifications of a metro
door management system in VDM-SL. The purpose of the presentation is
to describe alternatives to the Metro specification developed in the
SPECTRUM project (Esprit project no.\ 23173) and presented in the
Proceedings of FMSP'98, published by ACM SIGSOFT, March
1998\footnote{Also available at {\small
    ftp://ftp.ifad.dk/pub/papers/final.ps.gz}.}. The present document
is a draft and the specifications have not tested, so there may be
bugs!

Basically, the metro door management system can be described as
follows. There are three actors: the train driver, the train itself,
and the doors. The train driver has four button that he can use to
control the train and the doors: accelerate, break, open and
close. Our main concern is the safety requirement that the train
should not be able to move while the doors are open.

\section{A First Straightforward Specification}
The specification below models actions (or buttons) as VDM-SL
operations. The state of the doors and the train is held in a VDM-SL state.

\documentclass{article}
\usepackage{overture}%vdmsl-2e}

\title{Abstract Metro Specifications in VDM-SL}
\author{Sten Agerholm\\{\small IFAD, Forskerparken 10, DK-5230 Odense M,
  Denmark}\\{\small Email: sten@ifad.dk}}
\begin{document}
\maketitle

\section{Introduction}
Below I'll provide three different abstract specifications of a metro
door management system in VDM-SL. The purpose of the presentation is
to describe alternatives to the Metro specification developed in the
SPECTRUM project (Esprit project no.\ 23173) and presented in the
Proceedings of FMSP'98, published by ACM SIGSOFT, March
1998\footnote{Also available at {\small
    ftp://ftp.ifad.dk/pub/papers/final.ps.gz}.}. The present document
is a draft and the specifications have not tested, so there may be
bugs!

Basically, the metro door management system can be described as
follows. There are three actors: the train driver, the train itself,
and the doors. The train driver has four button that he can use to
control the train and the doors: accelerate, break, open and
close. Our main concern is the safety requirement that the train
should not be able to move while the doors are open.

\section{A First Straightforward Specification}
The specification below models actions (or buttons) as VDM-SL
operations. The state of the doors and the train is held in a VDM-SL state.

\documentclass{article}
\usepackage{overture}%vdmsl-2e}

\title{Abstract Metro Specifications in VDM-SL}
\author{Sten Agerholm\\{\small IFAD, Forskerparken 10, DK-5230 Odense M,
  Denmark}\\{\small Email: sten@ifad.dk}}
\begin{document}
\maketitle

\section{Introduction}
Below I'll provide three different abstract specifications of a metro
door management system in VDM-SL. The purpose of the presentation is
to describe alternatives to the Metro specification developed in the
SPECTRUM project (Esprit project no.\ 23173) and presented in the
Proceedings of FMSP'98, published by ACM SIGSOFT, March
1998\footnote{Also available at {\small
    ftp://ftp.ifad.dk/pub/papers/final.ps.gz}.}. The present document
is a draft and the specifications have not tested, so there may be
bugs!

Basically, the metro door management system can be described as
follows. There are three actors: the train driver, the train itself,
and the doors. The train driver has four button that he can use to
control the train and the doors: accelerate, break, open and
close. Our main concern is the safety requirement that the train
should not be able to move while the doors are open.

\section{A First Straightforward Specification}
The specification below models actions (or buttons) as VDM-SL
operations. The state of the doors and the train is held in a VDM-SL state.

\include{generated/latex/specification/metro.vdmsl}

\end{document}




\end{document}




\end{document}




\end{document}


