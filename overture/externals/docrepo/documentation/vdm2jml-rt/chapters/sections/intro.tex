\section{Introduction}
\label{sec:intro}

\ac{dbc} is an approach for designing software based on concepts such
as preconditions, postconditions and invariants~\cite{Meyer88}. These
concepts are referred to as ``\emph{contracts}'', according to a
conceptual metaphor for the conditions and obligations of a business
contract. An example of a formal method that uses \ac{dbc} elements is
\ac{vdm}, which was originally developed at IBM in Vienna for the
development of a compiler for
PL/1~\cite{Bjorner&78,Fitzgerald&08c,Fitzgerald&09}. One way to
realise a \ac{vdm} specification in a programming language is through
refinement~\cite{Woodcock&96}. This is a stepwise process by which one
can transform a formal model into a program that can be verified to
semantically satisfy its contracts~\cite{SDRA}.

Another way to realise a \ac{vdm} specification is using
code-generation. The idea is for the generated code to be a refinement
of the specification, but which is not achieved through step-wise
refinement, but rather in one step through code-generation translation
rules. Code-generation aims to reduce the resources needed to realise
the model as well as to avoid introducing problems in the
implementation due to manual translation of model into code. However,
current \ac{vdm} code-generators do not make any guarantees about the
correctness of the generated code, nor do they provide the necessary
means to help check that the code meets the specification. Naturally,
this casts doubt on the value of code-generation as a way to realise a
\ac{vdm} model, since the goal is to develop software that meets the
specification.

In this report we improve code-generation of \ac{vdm} models by
allowing the generated code to be checked against the system
properties described by the \ac{vdm} contracts. This helps ensure that
the generated code meets the \ac{vdm} specification, and is achieved
as described in this report.

Some \ac{dbc} technologies are tailored to specify detailed designs of
programming interfaces for a particular programming
language~\cite{Wing87}. An example of one such technology is
\ac{jml}~\cite{Burdy&05} -- a formal specification language that uses
\ac{dbc} elements, written as specialized comments, to specify the
behaviour of Java classes and interfaces. \ac{jml} annotations can be
analysed statically or checked dynamically using \ac{jml} tools.
Therefore, \ac{jml} can be seen as a technology that serves to bridge
the gap between an abstract system specification and its Java
implementation.

In this report we attempt to bridge this gap even further by proposing
a way to automatically translate a specification written in \ac{vdmsl}
to a \ac{jml}-annotated Java implementation. Current \ac{vdm}
code-generators either ignore or provide limited code-generation
support for the contract-based elements and type constraints of
\ac{vdm}. Ideally we should be able to preserve the contracts and type
constraints when the system specification is implemented, since (1)
they serve to document the intention and properties of the system and
(2) they can be used to check the system realisation for
correctness. Ensuring that the contracts and type constraints, as
originally specified in \ac{vdm}, hold for the system implementation
potentially requires many extra checks to be added to the code. Adding
these checks to the code manually is tedious and prone to
errors. Instead, these checks could be generated
automatically. Representing contracts and type constraints in \ac{jml}
also has the advantage that these checks may be ignored by the Java
compiler. This allows the system realisation to be executed without
the overhead of checking the contracts and type constraints, if
desired.

The two main contributions of our work are (1) a collection of
semantics-preserving rules for translating a \ac{vdmsl} specification
to a \ac{jml}-annotated Java program and (2) an implementation of
these rules as an extension to Overture's \cite{Larsen&10a,Overture}
\ac{vdm}-to-Java code-generator~\cite{Jorgensen&14a}.

The rules propose ways to translate the \ac{dbc} elements of
\ac{vdmsl} to \ac{jml} annotations; these annotations are added to the
Java code produced by Overture's Java code-generator. The rules cover
checking of preconditions, postconditions and invariants, but the
translator also produces \ac{jml} checks to ensure that no type
constraints are violated across the translation. We present the rules
one by one and demonstrate, using a case study model of an \ac{atm},
how the code-generator extension translates a \ac{vdmsl} specification
to \ac{jml}-annotated Java code.

Since the translation is not formally defined we have used the
OpenJML~\cite{Cok&11} runtime assertion checker to validate our work
--- in particular by generating \ac{jml} constructs supported by this
tool. More specifically, the \ac{jml} translator has been tested by
running examples through the tool in order to validate each of the
translation rules (see \autoref{sec:assess} for more details).

Following this section, we describe \ac{dbc} with VDM-SL and JML in
\autoref{sec:vdmjml}. We continue by presenting the implementation of
the \ac{jml} translator in \autoref{sec:sl2java}. Then we describe the
rules used to translate a \ac{vdmsl} specification to a
\ac{jml}-annotated Java program in sections \ref{sec:sl-dbc-gen} to
\ref{sec:sl-other-gen}. Next, we assess the correctness of the
translation in \autoref{sec:assess}. Finally we describe related work
in \autoref{sec:related} and present future plans and conclude in
\autoref{sec:con}.



%%% Local Variables:
%%% mode: latex
%%% TeX-master: "../../vdm2jml-tr"
%%% End:
