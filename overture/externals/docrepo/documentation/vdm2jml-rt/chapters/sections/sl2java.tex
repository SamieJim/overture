\section{The implementation of the \ac{jml} translator}
\label{sec:sl2java}

The \ac{jml} translator is implemented as an extension to Overture's
\ac{vdmsl}-to-Java code-generator, which provides code-generation
support for a large executable subset of \ac{vdm}. This section
describes how the \ac{jml} translator has been implemented, and
explains the details of the Java code-generator that are needed in
order to understand how the \ac{jml} translator works.

\subsection{The implementation}

The Java code-generator is developed using Overture's code-generation
platform -- a framework for constructing code-generators for
\ac{vdm}~\cite{Jorgensen&14a}. This platform is used by the Java
code-generator to parse the \ac{vdmsl} model sources and to construct
an \ac{ir} of the model -- an \ac{ast} that constitutes an internal
representation of the generated code. The Java code-generator uses the
code-generation platform to \emph{transform} the \ac{ir} into a tree
structure that eventually is translated directly into Java code. The
translation of the \ac{ir} into Java is handled by the code-generation
platform's code emission framework, which uses the Apache Velocity
template engine~\cite{ApacheVelocity}.

The Java code-generator exposes the \ac{ir} during the code-generation
process, which allows the \ac{jml} translator to intercept the
code-generation process and further transform the \ac{ir}. These
additional transformations are used to decorate the \ac{ir} with nodes
that contain the \ac{jml} annotations. Using the code emission
framework, the final version of the \ac{ir} is translated into a
\ac{jml}-annotated Java program.

The \ac{jml} translator is publicly available in Overture version
2.3.8 (as of July 2016) onwards~\cite{Overture}. Furthermore, the
\ac{jml} translator's source code is available via the Overture tool's
open-source code repository~\cite{OvertureGithub}.

\subsection{Overview of the translation}

In the generated code, a module is represented using a \kw{final} Java
class with a \kw{private} constructor, since \ac{vdmsl} does not
support inheritance and a module cannot be instantiated. Due to the
latter, both operations and functions are code-generated as
\kw{static} Java methods.

Module state is represented using a \kw{static} class field in the
module class to ensure that only a single state component exists at
any given time. The state component is represented using a record
value, and as a consequence, an additional record type is 
generated to represent it.

Each variable in \ac{vdmsl} is passed by value, i.e.\ as a \emph{deep
  copy}, when it is passed as an argument, appears on the right-hand
side of an assignment or is returned as a result. As a consequence,
aliasing can never occur in a \ac{vdmsl} model. Types are different in
Java, where objects are modified via object references or
pointers. Therefore different object references can be used to modify
the same object. To avoid such aliasing in the generated code, data
types are code-generated with functionality to support value type
behaviour.

Every record definition code-generates to a class definition with
accessor methods for reading and manipulating the fields. This class
implements \texttt{equals} and \texttt{copy} methods to support
comparison based on structural equivalence and deep copying,
respectively. In this way the call-by-value semantics of \ac{vdmsl}
can be preserved in the generated code by invoking the \texttt{copy}
method, which helps to prevent aliasing. Similarly the \texttt{equals}
method can be invoked to compare code-generated records based on
structural equivalence rather than comparing addresses of object
references. A record object can then be obtained by invoking the
constructor of the record class or by invoking the \texttt{copy}
method of an existing record object.

Java does not support the definition of aliases of existing types,
such as the \texttt{Amount} named invariant type
in~\autoref{lst:vdmAmount}. Therefore, the Java code-generator chooses
not to code-generate class definitions for these types. Instead, a use of
a named invariant type is replaced with its domain type (described
in \autoref{sub:vdmsl}). Since the named invariant type is an alias of
an existing type this is fine, as long as we make sure to check that
the type invariant holds.

% [I think the following may be more natural for Java -- Gary]
% [Yes you can argue that, but there are other (good) reasons for not doing it. I omitted this paragraph as I don't want the paper to get into a discussion about this -- Peter TJ]
% Alternatively, the Java code-generator could have generated a class
% definition to represent the named invariant type directly. As a
% consequence of the current approach fewer class definitions are
% generated. However, as we shall see in~\autoref{sec:sl-other-gen},
% this approach imposes some different challenges to the translation of
% named invariant types to \ac{jml} compared to that of translating
% state invariants.

To assist the translation of \ac{vdm} to Java, the existing Java
code-generator uses a runtime library, which among other things,
includes Java implementations for some of the different \ac{vdm} types
and operators. The \texttt{Tuple} class, for example, is used to
represent tuple types and enables construction of tuple values. Sets,
sequences and maps are represented using the \texttt{VDMSet},
\texttt{VDMSeq} and \texttt{VDMMap} classes, which themselves are
based on Java collections, and so on. The runtime library's collection
classes are used as raw types (e.g.\ \texttt{VDMSet}) in the generated
code, and therefore they are never passed a generic type argument. Raw
types provide a convenient way to represent \ac{vdm} collections that
store elements of some union type -- a kind of type that Java does not
support.

In addition to using the existing runtime library, the \ac{jml}
translator also contributes a small runtime library to aid the
generation of \ac{jml} checks. This runtime library, which we
subsequently refer to as \texttt{V2J}, is an extension of the existing
Java code-generator runtime library. As we shall see in
\autoref{sec:col}, the \texttt{V2J} runtime is mostly used in the
generated \ac{jml} checks to ensure that instances of collections
respect the \ac{vdm} types that produce them.


%%% Local Variables:
%%% mode: latex
%%% TeX-master: "../../vdm2jml-tr"
%%% End:
