\section{Conclusion and future plans}
\label{sec:con}

In this report we have demonstrated how \ac{vdmsl} models can be
translated to \ac{jml}-annotated Java programs that can be checked for
correctness using \ac{jml} tools. The \ac{jml} translator uses
\ac{jml} to represent the \ac{dbc} elements of \ac{vdmsl}, and
generates checks that help ensure the consistency of \ac{vdmsl} types
across the translation.

The principles for pre- and postconditions in \ac{vdmsl} and \ac{jml}
are similar although there are subtle semantic differences between the
two notations. These differences are mostly caused by the fact that
\ac{jml} is built on top of Java, where object types use reference
semantics. \ac{vdmsl}, on the other hand, solely uses value
types. Therefore, it is necessary to employ deep cloning principles
when representing value types in \ac{jml}-annotated Java code.

Checking state and record invariants in the generated code is
complicated due to two reasons: First, atomic execution in \ac{vdm}
requires a way to control when invariant checking must be done. We
achieve this by using a \kw{ghost} field to indicate when invariant
checking is enabled, and update it before entering and leaving the
\kw{atomic} statement. Secondly, we have demonstrated that transitive
dependencies between records sometimes require extra \ac{jml} checks
to be generated to ensure that the invariant checks are evaluated when
they should.

The differences between the type systems of \ac{vdmsl} and Java
further necessitate extra checks to be produced. These checks are
needed to ensure that the generated code does not violate any of the
constraints imposed by the types in the \ac{vdmsl} model. Overture
performs these dynamic type checks internally, whereas they must be
made explicit in Java.

Although \ac{dbc} languages often support many of the same \ac{dbc}
concepts, it is the semantic differences between the languages that
make developing a translation challenging. In this report we have shown
several examples of such differences and how they can be
addressed. Naturally, translating between other specification language
pairs may reveal other differences and design details that are of
interest to researchers and practitioners working on comparable
tasks. However, based on the experiences gained by developing the
\ac{vdmsl}-to-\ac{jml} translation, we list some of the design details
that we believe are likely to challenge the development of
translations between other specification language pairs:

\begin{description}

\item[Invariants:] The times when invariants are evaluated varies
  across specification languages. For example, in \ac{vdm} they have
  to hold at all times (except inside \kw{atomic} statements), whereas
  in \ac{jml} they must hold in visible states. When invariants have
  different semantics the translation must find a way to either
  produce or reduce the number of invariant checks at the appropriate
  places in the code.

\item[Type systems:] The differences between type systems require
  careful attention when developing a translation. Especially, when
  the destination language (e.g.\ \ac{jml}) uses a more
  ``coarse-grained'' type system than the source language (e.g.\
  \ac{vdmsl}). For such situations extra checks must be produced to
  ensure that types are used consistently across the translation. In
  our work we use the function \texttt{Is(v,T)} to produce these extra
  checks.

\item [Atomic execution:] Languages may use dedicated constructs to
  represent atomic execution (e.g.\ \ac{vdm}) or by allowing
  invariants not to hold at certain times (e.g.\ \ac{jml}). In this
  report an example was given of how a dedicated construct can be
  emulated in a language that does not support one natively.

\item [Old state:] Despite pre- and postconditions being similar
  concepts in different specification languages it is likely that the
  notion of old state may require careful handling when developing a
  translation between two specification languages. In our work, a deep
  cloning principle was employed to ensure the correct construction of
  the old state.

\end{description}

In the future we plan to use this work in the context of test
automation. In \ac{vdm} it is possible to specify a trace definition
in a way similar to that of a regular expression. This trace can then
be expanded into a large collection of tests that can be executed
against the model. This is a useful way to detect deficiencies in the
model, such as missing preconditions, postconditions and
invariants~\cite{Larsen&10c}.

We plan to code-generate the trace expansion such that the 
tests can be executed against the code-generated version of the
model. The work presented in this report can then be used to detect
contract or type violations and give verdicts to the code-generated
trace tests. We believe that this will be particularly advantageous
for execution of large collections of tests. We expect this approach
to significantly increase execution speed for test cases and also
allow more tests to be executed. In addition, we plan to look into
\ac{jml}-generation for other \ac{vdm} dialects such as
\ac{vdm}++. However, since \ac{vdm}++ is object-oriented and supports
concurrency, we envisage that this will give rise to a completely new
set of challenges not addressed by the work in this report.

So far the analysis of the generated Java/\ac{jml} has primarily been
limited to runtime assertion checking. Another item of future work is
to formally verify the generated code against the \ac{jml}
specification. In particular, by investigating to what extent this is
possible, and whether the \ac{jml} translation can be optimised in a
way that better supports formal verification through static
analysis. For example, currently the translation produces auxiliary
methods for invariants and pre- and postconditions that are used as
part of the \ac{jml} specification. However, use of method calls in
specifications complicates static analysis due to, for example, the
possibility of exceptions or non-terminating behaviour~\cite{Cok05}.
  
We hope that our work will serve as inspiration for other researchers
who seek to bridge the gap between other specification notations and
implementation technologies that support the \ac{dbc} approach. We
believe that the rules proposed in this report can be useful for others
who want to translate between specification languages such as ASM, B
and Z and implementation technologies such as Spec\#, Sparc-Ada and
Eiffel.


%%% Local Variables:
%%% mode: latex
%%% TeX-master: "../../vdm2jml-tr"
%%% End:
