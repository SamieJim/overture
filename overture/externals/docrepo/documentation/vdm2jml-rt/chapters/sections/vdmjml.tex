\section{\ac{dbc} with \ac{vdmsl} and \ac{jml}}
\label{sec:vdmjml}

In this section we describe \ac{vdmsl} and \ac{jml}. We cover
different types and all the contract-based elements of \ac{vdmsl},
focusing specifically on the \ac{vdm}-10 release, which we are
targeting in our work. The \ac{jml} constructs described in this
section cover those that are used to implement the translation rules.

\subsection{\ac{vdmsl}}
\label{sub:vdmsl}

\ac{vdmsl} is an ISO standardised sequential modelling language that
supports description of data and functionality. The ISO standard has
later been informally extended with modules to allow type definitions,
values (constants) and functionality to be imported and exported
between modules. A module may define a single state component, which
can be constrained by a state invariant. State is modified by
assigning a new value to a state designator, which can be either a
name, a field reference or a map or sequence reference, as described
in the \ac{vdm} language reference manual~\cite{Larsen&10b}.

Module state, if specified, implicitly defines a record type, which is
tagged with the state name and also defines the type of the state
component. The state type can be used like any other record type
explicitly defined by the modeller -- the difference being that the
state invariant~\cite{Andrews&98} constrains the state type and thus
every instance of this record type.

Data are defined by means of built-in basic types covering, for
instance, numbers, booleans, quote types and characters. A quote type
corresponds to an enumerated type in a language such as Pascal. The
basic types can be used to form new structured data types using
built-in type constructors that support creation of union types, tuple
types and record types. A type may also be declared optional, which
allows \kw{nil} to be used to represent the absence of a value. For
collections of values, \ac{vdmsl} supports sets, sequences and
maps. The built-in data types, type constructors and collections can
be used to form named user defined types, which can be constrained by
invariants. We refer to these types as \emph{named invariant
  types}. As an example, \autoref{lst:vdmAmount} shows the definition
of the named invariant type \texttt{Amount}, which is used to
represent an amount of money deposited or withdrawn by an account
holder. This type is defined based on natural numbers (excluding
zero), i.e.\ the built-in basic type \kw{nat1} in \ac{vdmsl}. For this
particular example, we say that \kw{nat1} is the \emph{domain type} of
\texttt{Amount}. We further constrain \texttt{Amount} using an
invariant, by requiring a value of this type to be less than
2000. Specifically, for the invariant shown in
\autoref{lst:vdmAmount}, the \texttt{a} on the left-hand side of the
equality is a pattern that matches values of type
\texttt{Amount}. This pattern is used to express the invariant
predicate for this named invariant type.

\begin{vdmsl}[style=customVdm,caption={Example of a \ac{vdmsl} named invariant type.},label={lst:vdmAmount}]
types
Amount = nat1
inv a == a < 2000;
\end{vdmsl}

\subsubsection{Functional descriptions}
\label{sec:functional-descriptions}

In \ac{vdm}, functionality can be defined in terms of functions and
operations over data types with a traditional call-by-value semantics.
Functions are referentially transparent and therefore they are not
allowed to access or manipulate state directly, whereas operations
are. Therefore, a function cannot call an operation.\footnote{With the
  recent introduction of \kw{pure} operations into \ac{vdm}-10 (not to
  be confused with \kw{pure} methods in \ac{jml}) it has become
  possible to invoke operations, albeit \kw{pure} ones, from a
  function. This feature was introduced to address issues with the
  object-oriented dialect of \ac{vdm}, called \ac{vdm}++, but was made
  available in every \ac{vdm}-10 dialect (including \ac{vdmsl}).} In
addition to accessing module state, operations may also use the
\kw{dcl} statement to declare local state designators which can be
assigned to. Subsequently the term \emph{functional description} will
be used to refer to both functions and operations. As an example, a
function that uses the \texttt{MATH`sqrt} library function to
calculate the square root of a real number is shown in
\autoref{lst:vdmSqrt}.

\begin{vdmsl}[style=customVdm,caption={\ac{vdmsl} function for
calculating the square root of a number.},label={lst:vdmSqrt}]
sqrt :  real -> real
sqrt (x) == MATH`sqrt(x)
pre x >= 0
post RESULT * RESULT = x;
\end{vdmsl}

Functional descriptions can be implicitly defined in terms of pre- and
postconditions, which specify conditions that must hold before and
after invoking the functional description. Alternatively, a functional
description can be \emph{explicitly} defined by means of an algorithm,
as shown in \autoref{lst:vdmSqrt}. The \ac{jml} translator supports
both implicitly and explicitly defined functional
descriptions. However, only methods that originate from explicitly
defined functional descriptions can be executed.

The precondition of a function can refer to all the arguments of the
function it guards. The same applies to the postcondition of a
function, which can also refer to the result of the execution using
the reserved word \kw{RESULT}. For the square root function in
\autoref{lst:vdmSqrt} we require that the input is a positive number
(the precondition), and that the square of the function result equals
the input value (the postcondition).

Function definitions are derived for the pre- and postconditions of
\texttt{sqrt} from \texttt{sqrt}'s \kw{pre} and \kw{post}
clauses. These function definitions do not appear in the model, but
they are used internally by the Overture interpreter to check for
contract violations. However, to clarify, the pre- and postcondition
functions of \texttt{sqrt} are shown in
\autoref{lst:vdmFuncPrePost}. In this listing, \texttt{+>} specifies
that \texttt{pre\_sqrt} and \texttt{post\_sqrt} are total functions,
and not partial functions, which use the \texttt{->} type constructor.

\begin{vdmsl}[style=customVdm,caption={Pre- and postcondition
functions for the \texttt{sqrt} function shown in
\autoref{lst:vdmSqrt}.},label={lst:vdmFuncPrePost}]
pre_sqrt:real +> bool
pre_sqrt(x) == x >= 0;

post_sqrt:real*real +> bool
post_sqrt(x,RESULT) == RESULT * RESULT = x;
\end{vdmsl}

Similarly, the pre- and postcondition functions of an operation are
also derived. To demonstrate this, consider the \texttt{inc} operation
in \autoref{lst:vdmInc}. This operation takes a real number as input,
adds it to a counter (defined using a state designator), and returns
the new counter value. In this listing \texttt{counter\~} and
\texttt{counter} refer to the counter values before and after the
operation has been invoked, respectively.

\begin{vdmsl}[style=customVdm,caption={\ac{vdmsl} operation for
incrementing a counter.},label={lst:vdmInc}]
inc : real ==> real
inc (i) == (
 counter := counter + i;
 return counter;
)
pre i > 0
post counter = counter~ + i and
     RESULT = counter;
\end{vdmsl}

A precondition of an operation can refer to the state, \texttt{s},
before executing the operation, whereas the postcondition of an
operation can read both the before and after states. State access is
achieved by passing copies of the state to the pre- and postcondition
functions. The corresponding pre- and postcondition functions for
\texttt{inc} are shown in \autoref{lst:vdmOpPrePost} where the
parameters \texttt{s\~} and \texttt{s} of \texttt{post\_inc} refer to
the state (that contains the counter value) before and after execution
of \texttt{inc}. We further use \texttt{S} to denote the record type
that represents the module's state.

\begin{vdmsl}[style=customVdm,caption={Pre- and postcondition
functions for the \texttt{inc} operation shown in
\autoref{lst:vdmInc}.},label={lst:vdmOpPrePost}]
pre_inc:real*S +> bool
pre_inc(i,s) == i > 0;

post_inc:real*real*S*S +> bool
post_inc(i,RESULT,s~,s) ==
s.counter = s~.counter + i and
RESULT = s.counter;
\end{vdmsl}

The function descriptions in \autoref{lst:vdmOpPrePost} assume that
the pre- and postconditions are defined (using the \kw{pre} and
\kw{post} clauses) and that the state of the module enclosing the
functional description exists. For the cases where pre- and
postconditions are not defined they can be thought of as functions
that yield \kw{true} for every input. Furthermore, when no state
component is defined, the pre- and postcondition functions simply omit
the state parameters. Similarly, when an operation does not return a
result (it specifies void as the return type) the postcondition
function omits the \kw{RESULT} parameter.

For each type definition constrained by an invariant, such as
\texttt{Amount} shown in \autoref{lst:vdmAmount}, a function is
implicitly created to represent the invariant -- see
\autoref{lst:vdmTypeInv}. The Overture tool uses this function
internally to check whether a value is consistent with respect to a
given type (e.g.\ \texttt{Amount})~\cite{Lausdahl&11}. Note that since
all invariants are functions they are not allowed to depend on state
of other modules. Specifically, invariants can only invoke functions
and access global constants (possibly defined in other modules).

\begin{vdmsl}[style=customVdm,caption={Invariant function for type
definition \texttt{Amount}.},label={lst:vdmTypeInv}]
inv_Amount : Amount +> bool
inv_Amount (a) == a < 2000;
\end{vdmsl}

\subsubsection{Atomic execution}

Multiple consecutive statements are sometimes needed to update the
state designators to make them consistent with the system's
invariants. For example, assume that we have a system that uses two
state designators called \texttt{evenID\sub{1}} and
\texttt{evenID\sub{2}} to store even and different numbers. For this
example, we will assume that these state designators are of type
\texttt{Even} -- a type that constrains these state designators to
store even numbers. To help ensure that the uniqueness constraint (a
state invariant) is not violated during an update, multiple
assignments can be grouped in an \kw{atomic} statement block as shown
in~\autoref{lst:vdmAtomic}. Given the type \texttt{Even} of the state designators
\texttt{evenID\sub{1}} and \texttt{evenID\sub{2}} it is as if the
atomic statement is evaluated as shown in \autoref{lst:vdmAtomicEval}.

\begin{vdmsl}[style=customVdm,caption={Atomic update in
\ac{vdm}.},label={lst:vdmAtomic}]
atomic (
 evenID!\sub{1}! := exp!\sub{1}!;
 evenID!\sub{2}! := exp!\sub{2}!;
)
\end{vdmsl}


\begin{vdmsl}[style=customVdm,caption={The execution semantics of the
\kw{atomic} statement.},label={lst:vdmAtomicEval}]
let t!\sub{1}! : Even = exp!\sub{1}!,
    t!\sub{2}! : Even = exp!\sub{2}!
in (
 -- Turn off invariants
 evenID!\sub{1}! := t!\sub{1}!;
 evenID!\sub{2}! := t!\sub{2}!;
 -- Turn on invariants
 -- Check invariants hold
);
\end{vdmsl}

Executing the \kw{atomic} statement block is semantically equivalent
to first evaluating the right-hand sides of all the assignments before
turning off invariant checks, and then binding the results to the
corresponding state designators. After all the assignments have been
executed, it must be ensured that all invariants hold.

There are three properties that follow from the evaluation semantics
of the \kw{atomic} statement block that are worth mentioning:

\begin{enumerate}

\item When evaluating the right-hand sides of the assignment
  statements, potential contract violations will be reported.

\item Temporary identifiers, used to store the right-hand side
  results, are explicitly typed and therefore violations of named
  invariant types for these variables will be reported. The explicit
  type annotations thus ensure that the right-hand side of a state
  designator assignment is checked to be consistent with the type of
  said state designator.

\item Assignment statements cannot see intermediate values of state
  designators.

\end{enumerate}

\subsection{\ac{jml}}
\label{sec:jml}

Although \ac{jml}~\cite{Leavens&13} is designed to specify arbitary
sequential Java programs, in this subsection we only describe the
features needed for the translation from \ac{vdmsl}.

A method specified with the \kw{pure} modifier in \ac{jml} is not
permitted to have write effects; such 
methods are allowed to be used in specifications. Pure methods are
used to translate \ac{vdmsl} functions.

A class invariant in \ac{jml} should hold whenever the non-helper
methods of that class are not being executed; thus invariants must
hold in each method's before and after states. However, a method
declared with the \kw{helper} annotation in a type \texttt{T} does not
have its pre- and postconditions augmented with \texttt{T}'s
invariants.  Helper methods (and constructors) must either be pure or
private~\cite{Leavens&13}, so that the invariant will hold at the
beginning and end of all client-visible
methods~\cite{Mueller-Poetzsch-Heffter-Leavens06}. The before and
after states of non-helper methods and constructors are said to be
\emph{visible states}; thus invariants must hold in all visible
states.  \ac{jml} distinguishes between instance and static
invariants. An \emph{instance} invariant can refer to the non-static
(i.e.\ instance) fields of an object.  A \emph{static\/} invariant
cannot refer to an object's non-static fields; thus static invariants
are used to specify properties of static fields.

An assertion can reference the invariant for an object explicity using
a predicate of the form \invariantfor{\texttt{(e)}}, which is
equivalent to the invariant for \texttt{e}'s static type \cite[section
12.4.22]{Leavens&13}.

In \ac{jml} pre- and postconditions are written using the keywords
\kw{requires} and \kw{ensures}, respectively. In the specification of
a postcondition, one writes \texttt{\textbackslash \textbf{old}(e)} to
refer to the before state value of an expression \texttt{e}.  For
example, an \texttt{increment} method that writes a field
\texttt{count} could be specified as shown in
\autoref{lst:jmlCounMethod}.

\begin{lstlisting}[style=customJml,label={lst:jmlCounMethod},caption={Example of a \ac{jml} specification for a Java method.}]
//@ requires count < Integer.MAX_VALUE;
//@ modifies count;
//@ ensures count == \old(count)+1;
void increment() {
  count++;
}
\end{lstlisting}
Method postconditions may also use the keyword {\RESULT} to refer to
the value returned by the method.

Specification expressions in \ac{jml} can use Java expressions that
are pure (have no write effects), and also some logical operators,
such as implication \texttt{==>}, and quantifiers such as {\JMLforall}
and {\JMLexists}.

In addition to method pre- and postconditions, one can also write
assertions anywhere a Java statement can appear, using \ac{jml}'s
\kw{assert} keyword. Such assertions must hold whenever they are executed.

One way to specify the abstract state of a class is to use \ac{jml}'s
\kw{ghost} variables. Ghost variables are specification-only variables
and fields of objects that can only be used in \ac{jml} specifications
and in \ac{jml} \kw{set} statements. A set statement is an assignment
statement whose target is a ghost variable.

By default, \ac{jml} variables and fields may not hold the \kw{null}
value. However, should one wish to specify that all fields of a class
may hold \kw{null}, then one can annotate the class's declaration with
\kw{nullable\_by\_default}.


%%% Local Variables:
%%% mode: latex
%%% TeX-master: "../../vdm2jml-tr"
%%% End:
