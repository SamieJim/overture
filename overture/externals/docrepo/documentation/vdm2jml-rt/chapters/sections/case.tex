\section{Case study example}
\label{sec:case}

Throughout the report we will demonstrate the translation rules using a
case study model of an \ac{atm}. The model consists of a single
module, \texttt{ATM} (shown in \autoref{lst:vdmCase}), which uses a
state definition to record information about

\begin{itemize}

\item The debit cards considered valid by the system (named
  \texttt{validCards}).

\item The debit card currently inserted into the \ac{atm}, if any
  (\texttt{currentCard}).

\item If a valid PIN code has been entered (\texttt{pinOk}) for the
  debit card currently inserted into the \ac{atm} and,

\item all the bank accounts known to the system (\texttt{accounts}).

\end{itemize}

\begin{vdmsl}[style=customVdm,caption={\ac{vdmsl} module representing
an \ac{atm}.},label={lst:vdmCase}]
module ATM
definitions
state St of
 validCards : set of Card
 currentCard : [Card]
 pinOk : bool
 accounts : map AccountId to Account
 init St == St = @mk_@St({},nil,false,{|->})
 inv @mk_@St(v,c,p,a) ==
  (p or c <> nil => c in set v)
  and
  forall id1, id2 in set dom a &
   id1 <> id2 =>
   a(id1).cards inter a(id2).cards = {}
end
 ...
operations
GetStatus : () ==> bool * seq of char
GetStatus () == ...

OpenAccount : set of Card * AccountId ==> ()
OpenAccount (cards,id) == ...

AddCard : Card ==> ()
AddCard (c) == ...

RemoveCard : Card ==> ()
RemoveCard (c) == ...

InsertCard : Card ==>
  <Accept>|<Busy>|<Reject>  
InsertCard (c) == ...

EnterPin : Pin ==> ()
EnterPin (pin) == ...

ReturnCard : () ==> ()
ReturnCard () == ...

Withdraw : AccountId * Amount ==> real
Withdraw (id, amount) == ...

Deposit : AccountId * Amount ==> real
Deposit (id, amount) == ...
end
\end{vdmsl}

For simplicity, \autoref{lst:vdmCase} omits type definitions and only
shows the state definition (including the state invariant) and the
signatures for some of the operations. The state invariant, shown in
\autoref{lst:vdmCase}, requires that at all times the following two
conditions must be met: a debit card must at most be associated with a
single account and secondly, for a PIN code to be considered valid,
the debit card currently inserted into the \ac{atm} must itself be a
valid debit card.

When the \ac{atm} model is translated to a \ac{jml}-annotated Java
program it can be checked for correctness using \ac{jml} tools. To
demonstrate this, consider the example in \autoref{lst:jmlScenario},
which creates a debit card, inserts it into the \ac{atm}, and performs
a transaction scenario.

\begin{lstlisting}[style=customJml,caption={Java code demonstrating
use of the implementation of the \ac{atm}
model.},label={lst:jmlScenario}]
Card c = new Card(5,1234);
// atm.ATM.AddCard(c); (missing statement)
atm.ATM.InsertCard(c);
atm.ATM.EnterPin(1234);
System.out.println(atm.ATM.GetStatus());
/* Transaction related code omitted */
atm.ATM.ReturnCard();
\end{lstlisting}

If this program is executed using the OpenJML runtime assertion
checker the output in \autoref{lst:racOutput} is reported.

\begin{lstlisting}[style=racOutput,caption={Inconsistent use of the
system detected using the OpenJML runtime assertion
checker.},label={lst:racOutput}]
Exception in thread "main" java.lang.AssertionError: Main.java:12: JML precondition is false
        atm.ATM.EnterPin(1234);
                        ^
atm/ATM.java:276: Associated declaration: Main.java:12: 
  //@ requires pre_EnterPin(pin,St);
      ^
	at Main.main(Main.java:17)
\end{lstlisting}

For this particular example, this error is reported because the debit
card \texttt{c} is not recognised as a valid debit card by the
system. Specifically, the scenario did not invoke
\texttt{atm.ATM.AddCard(c)} immediately after creating the debit
card. The return value of the \texttt{Insert} method did indicate that
the debit card was rejected, but this value was mistakenly discarded
in \autoref{lst:jmlScenario}. The error is reported by the runtime
assertion checker because entering a PIN code when no debit card is
inserted into the \ac{atm} is considered an error. After changing the
example in \autoref{lst:jmlScenario} to add \texttt{c} as a valid
debit card, no problems are detected by the runtime assertion checker,
as expected. Therefore, the code executes as if it was compiled using
a standard Java compiler and executed on a regular Java virtual
machine. More, specifically, the system will report the status as
shown in \autoref{lst:racOutputSuccess} to indicate that the \ac{atm}
is not awaiting a debit card, and that a transaction is in progress.

\begin{lstlisting}[style=racOutput,caption={System output after fixing
the problem in
\autoref{lst:jmlScenario}.},label={lst:racOutputSuccess}]
mk_(false, "transaction in progress.")
\end{lstlisting}

As we proceed, in \autoref{sec:sl-dbc-gen} and
\autoref{sec:sl-types-gen} we elaborate on the specifics of each
\ac{vdm} definition in the case study model and demonstrate the
translation to \ac{jml}-annotated Java.


%%% Local Variables:
%%% mode: latex
%%% TeX-master: "../../vdm2jml-tr"
%%% End:
