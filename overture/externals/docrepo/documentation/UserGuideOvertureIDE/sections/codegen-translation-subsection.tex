The priority of the translation strategy is to remain faithful to the VDM-RT semantics described above.
%
The strategy therefore assumes that VDM-RT specifications have been validated using Overture's various facilities.
%
This section describes the strategy and the two sides of the translation mechanism, the implementation of the strategy and the native C support library.
%
The generated code can be compiled and executed by means of a CMake build mechanism and an empty \texttt{main.c} file, but no calls into the generated code proper are made.
%
%
%
%\paragraph{Executable VDM}
%\revisit{What is, but focus on what is not, executable.}
%
%
%
\subsubsection{Native Support Library}
Implementations generated from VDM-RT models consist of two parts, the generated code and a native support library\footnote{The design of the native library is based on the following four sources:\\\url{http://www.pvv.ntnu.no/~hakonhal/main.cgi/c/classes/}, accessed 2016-09-22.\\\url{http://www.eventhelix.com/RealtimeMantra/basics/}\\\parbox{1cm}{~}\url{ComparingCPPAndCPerformance2.htm}, accessed 2016-09-22.\\\url{http://www.go4expert.com/articles/}\\\parbox{1cm}{~}\url{virtual-table-vptr-multiple-inheritance-t16616/}, accessed 2016-09-22.\\\url{http://www.go4expert.com/articles/virtual-table-vptr-t16544/}, accessed 2016-09-22.}.
%
The native library is fixed and does not change during the code generation process.
%
We illustrate its design here by means of very simple generated VDM models.

The native library provides a single fundamental data structure in support of all the VDM-RT data types, called \texttt{TypedValue}.
%
The complete definition is shown in Listing \ref{lst:tvp}.
%
A pointer to \texttt{TypedValue} is \texttt{\#define}d as \texttt{TVP}, and is used throughout the implementation.
%
%
%
\begin{lstlisting}[language=C,caption={Fundamental code generator data type.},label={lst:tvp},frame=tlbr]{lst:vdmtype}
typedef enum {
	VDM_INT, VDM_NAT, VDM_NAT1, VDM_BOOL, VDM_REAL, 
	VDM_RAT, VDM_CHAR, VDM_SET, VDM_SEQ, VDM_MAP,
	VDM_PRODUCT, VDM_QUOTE, VDM_RECORD, VDM_CLASS
} vdmtype;

typedef union TypedValueType {
	void* ptr;			// VDM_SET, VDM_SEQ, VDM_CLASS,
	            			// VDM_MAP, VDM_PRODUCT
	int intVal;			// VDM_INT and INT1
	bool boolVal;			// VDM_BOOL
	double doubleVal;		// VDM_REAL
	char charVal;			// VDM_CHAR
	unsigned int uintVal;		// VDM_QUOTE
} TypedValueType;

struct TypedValue {
	vdmtype type;
	TypedValueType value;
};

struct Collection {
	struct TypedValue** value;
	int size;
};
\end{lstlisting}
%
%
%
An element of this type carries information about the type of the VDM value represented and the value proper.
%
For space efficiency, the value storage mechanism is a C \texttt{union}.

Members of the basic, unstructured types \texttt{int}, \texttt{char}, \emph{etc.}\@ are stored directly as values in corresponding fields.
%
Due to subtype relationships between certain VDM types, for instance \texttt{nat} and \texttt{nat1}, fields in the union can be reused.
%
Functions to construct such basic values are provided:
%
%
%
\begin{itemize}
\item \texttt{TVP newInt(int)}
\item \texttt{TVP newBool(bool)}
\item \texttt{TVP newQuote(unsigned int)}
\item \emph{etc.\@}
\end{itemize}
%
%
%
All the operations defined by the VDM language manual on basic types are implemented one-to-one.
%
They can be found in the header file\\\texttt{VdmBasicTypes.h}.

Members of structured VDM types, such as \texttt{seq} and \texttt{set}, are stored as references, owing to their variable size.
%
The \texttt{ptr} field is dedicated to these.
%
These collections are represented as arrays of \texttt{TypedValue} elements, wrapped in the C structure \texttt{Collection}.
%
The field \texttt{size} of \texttt{Collection} records the number of elements in the collection.
%
Naturally, collections can be nested.
%
At the level of VDM these data types are immutable and follow value semantics.
%
But internally they are constructed in various ways.
%
For instance, internally creating a fresh set from known values is different from constructing one value by value according to some filter on values.
%
In the former case a new set is created in one shot, whereas in the latter an empty set is created to which values are added.
%
Several functions are provided for constructing collections which accommodate these different situations.
%
%
%
\begin{itemize}
\item \texttt{newSetVar(size\_t, ...)}
\item \texttt{newSetWithValues(size\_t, TVP*)}
\item \texttt{newSeqWithValues(size\_t, TVP*)}
\item \emph{etc.\@}
\end{itemize}

These rely on two functions for constructing the inner collections of type \texttt{struct Collection} at field \texttt{ptr}:
%
%
%
\begin{itemize}
\item \texttt{TVP newCollection(size\_t, vdmtype)}
\item \texttt{TVP newCollectionWithValues(size\_t, vdmtype, TVP*)}
\end{itemize}
%
%
%
The former creates an empty collection that can be grown as needed by memory re-allocation.
%
The latter wraps an array of values for inclusion in a \texttt{TVP} value of structured type.
%
All the operations defined in the VDM language manual on structured types are implemented one-to-one.
%
They can be found in the header files \texttt{VdmSet.h}, \texttt{VdmSeq.h} and \texttt{VdmMap.h}.
%
%The functions implementing the operations defined in the VDM language manual \ref{} on basic and structured types are listed in Table \ref{}.
%
%

%\begin{tabular}{l|l|l}
%\textbf{Category} & \textbf{VDM Operation} & \textbf{Native Library Operation}\\
%\hline\\
%Basic type & & \\
%\hline\\
%Set & & \texttt{vdmSetMemberOf}\\
%\hline\\
%Sequence & & \\
%\hline\\
%Map & &
%\end{tabular}
%
VDM's object orientation features are fundamentally implemented in the native library using C \texttt{struct}s.
%
In brief, a class is represented by a \texttt{struct} whose fields represent the fields of the class.
%
The functions and operations of the class are implemented as functions associated with the corresponding \texttt{struct}.

Consider the following example VDM specification.
%
%
%
\begin{lstlisting}[language=VDM++,caption={Example VDM model.},label={lst:vdmexample},frame=tlbr]{lst:vdmexample}
class A
instance variables
	private i : int := 1;

operations
	public op : () ==> int
	op() ==
		return i;
end A
\end{lstlisting}
%
%
%
The code generator produces the two files \texttt{A.h} and \texttt{A.c}, shown below.
%
%
%
\begin{lstlisting}[language=C,caption={Corresponding header file \texttt{A.h}.},label={lst:vdmexheader},frame=tlbr]{lst:vdmexheader}
#include "Vdm.h"
#include "A.h"

#define CLASS_ID_A_ID 0

#define ACLASS struct A*

#define CLASS_A__Z2opEV 0

struct A
{
	VDM_CLASS_BASE_DEFINITIONS(A);
	 
	VDM_CLASS_FIELD_DEFINITION(A,i);
	
};

TVP _Z1AEV(ACLASS this_);

ACLASS A_Constructor(ACLASS);

\end{lstlisting}
%
%
%
The basic construct is a \texttt{struct} containing the class fields and the class virtual function table:
%
%
%
\begin{lstlisting}[language=C,caption={Macro for defining class virtual function tables.},label={lst:virttablmacro},frame=tlbr]{lst:virttablmacro}
#define VDM_CLASS_FIELD_DEFINITION(className, name) \
	TVP m_##className##_##name
	
#define VDM_CLASS_BASE_DEFINITIONS(className) \
	struct VTable * _##className##_pVTable; \
	int _##className##_id; \
	unsigned int _##className##_refs
\end{lstlisting}
%
%
%
The virtual function table contains information necessary for resolving a function call in a multiple inheritance context as well as a field which receives a pointer to the implementation of the operation \texttt{op}.
%
%
%
\begin{lstlisting}[language=C,caption={Virtual function table.},label={lst:virttablmacro},frame=tlbr]{lst:virttabl}
typedef TVP (*VirtualFunctionPointer)(void * self, ...);

struct VTable
{
	//Fields used in the case of multiple inheritance.
	int d;
	int i;
	
	VirtualFunctionPointer pFunc;
};
\end{lstlisting}
%
%
%
The rest of the important parts of the implementation consists of the function implementing \texttt{op()}, the definition of the virtual function table into which this slots and the complete constructor mechanism.
%
%
%
\begin{lstlisting}[language=C,caption={Corresponding implementation file \texttt{A.c}.},label={lst:vdmeximpl},frame=tlbr]{lst:vdmeximpl}
#include "A.h"
#include <stdio.h>
#include <string.h>

void A_free_fields(struct A *this)
{
	vdmFree(this->m_A_i);
}

static void A_free(struct A *this)
{
	--this->_A_refs;
	if (this->_A_refs < 1)
	{
		A_free_fields(this);
		free(this);
	}
}
 
/* A.vdmrt 6:9 */
static  TVP _Z2opEV(ACLASS this)
{
	/* A.vdmrt 8:10 */
	TVP ret_1 = vdmClone(newBool(true));

	/* A.vdmrt 8:3 */
	return ret_1;
}


static struct VTable VTableArrayForA [] =
{
	{0,0,((VirtualFunctionPointer) _Z2opEV),},
};
 

ACLASS A_Constructor(ACLASS this_ptr)
{
	if(this_ptr==NULL)
	{
		this_ptr = (ACLASS) malloc(sizeof(struct A));
	}

	if(this_ptr!=NULL)
	{
		this_ptr->_A_id = CLASS_ID_A_ID;
		this_ptr->_A_refs = 0;
		this_ptr->_A_pVTable=VTableArrayForA;

		this_ptr->m_A_i= NULL ;
	}

	return this_ptr;
}

// Method for creating new "class"
static TVP new()
{
	ACLASS ptr=A_Constructor(NULL);

	return newTypeValue(VDM_CLASS,
		(TypedValueType)
			{.ptr=newClassValue(ptr->_A_id,
				&ptr->_A_refs,
				(freeVdmClassFunction)&A_free,
				ptr)});
}
 

/* A.vdmrt 1:7 */
TVP _Z1AEV(ACLASS this)
{
	TVP __buf = NULL;

	if(this == NULL)
	{
		__buf = new();

		this = TO_CLASS_PTR(__buf, A);
	}

	return __buf;
}
\end{lstlisting}
%
%
%
\texttt{TO\_CLASS\_PTR} merely unwraps values and can be ignored for now.

Construction of an instance of class \texttt{A} starts with a call to \texttt{\_Z1AEV}.
%
An instance of \texttt{struct A} is allocated and its virtual function table is populated with the pointer to the implementation of \texttt{op()}, \texttt{\_Z2opEV}.
%
The latter name is a result of a name mangling scheme implemented in order to avoid name clashes in the presence of inheritance\footnote{The name mangling scheme is based on the following sources:\\\url{https://en.wikipedia.org/wiki/Name_mangling}, accessed 2016-09-28.\\\url{http://www.avabodh.com/cxxin/namemangling.html}, accessed 2016-09-28.}.
%
A header file called \texttt{MangledNames.h} provides the mappings between VDM model identifiers and mangled names in the generated code.
%
This mapping aids in writing the \texttt{main} function.
%
The scheme used is \texttt{ClassName\_identifier}.
%
Listing \ref{lst:manglednames} shows the contents of the file for the example model.
%
%
%
\begin{lstlisting}[language=C,caption={File \texttt{MangledNames.h}.},label={lst:manglednames},frame=tlbr]{lst:manglednames}
#define A_op _Z2opEV
#define A_A _Z1AEV
\end{lstlisting}
%
%
%

By default, the code generation process provides an empty \texttt{main.c} file such that it is possible to compile the generated code initially.
%
It will, of course, be completely inert.
%
The following example populated \texttt{main.c} file illustrates how to make use of the generated code.
%
%
%
\begin{lstlisting}[language=C,caption={Example \texttt{main.c} file.},label={lst:mainexample},frame=tlbr]{lst:mainexample}
#include "A.h"

int main()
{
	TVP a_instance = _Z1AEV(NULL);
	TVP result;
	
	result = CALL_FUNC(A, A, a_instance, CLASS_A__Z2opEV);
	
	printf("Operation op returns:  %d\n", result->value.intVal);
	
	vdmFree(result);
	vdmFree(a_instance);
	
	return 0;
}
\end{lstlisting}
%
%
%
Had the class \texttt{A} contained any \texttt{value}s or \texttt{static} fields, the very first calls into the model would have been to \texttt{A\_\allowbreak{}const\_\allowbreak{}init()} and \texttt{A\_\allowbreak{}static\_\allowbreak{}init()}.
%
As this is not the case here, an instance of the class implementation is first created, together with a variable to store the result of \texttt{op}.
%
The macro \texttt{CALL\_FUNC} carries out the necessary calculations for calling the correct version of \texttt{\_Z2opEV} in the presence of inheritance and overriding (not the case here).
%
%
%
\begin{lstlisting}[language=C,caption={Macros supporting function calls.},label={lst:fncallmacros},frame=tlbr]{lst:fncallmacros}
#define GET_STRUCT_FIELD(tname,ptr,fieldtype,fieldname) \
	(*((fieldtype*)(((unsigned char*)ptr) + \
			offsetof(struct tname, fieldname))))


#define GET_VTABLE_FUNC(thisTypeName,funcTname,ptr,id) \
	GET_STRUCT_FIELD(thisTypeName,ptr,struct VTable*, \
			_##funcTname##_pVTable)[id].pFunc


#define CALL_FUNC(thisTypeName,funcTname,classValue,id, args... ) \
	GET_VTABLE_FUNC( thisTypeName, \
			funcTname, \
			TO_CLASS_PTR(classValue,thisTypeName), \
			id) \
	(CLASS_CAST(TO_CLASS_PTR(classValue,thisTypeName), \
		thisTypeName, \
		funcTname), ## args)
\end{lstlisting}
%
%
%
The result is assigned to \texttt{result}, which is then accessed according to the structure of \texttt{TVP}.
%
The function \texttt{vdmFree} is the main memory cleanup function for variables of type \texttt{TVP}.
%
%
%
\subsubsection{Translating Features of VDM-SL}
In this section we discuss how the basic features of VDM-RT, those contained in the subset VDM-SL, are translated to C.
%
\paragraph{Basic data types}
Instances of the fundamental data types of VDM-SL (integers, reals, characters \emph{etc}.\@) translate directly to instances of type \texttt{TVP} with the appropriate field of the union structure \texttt{TypedValueType} set to the value of the instance.
%
They are instantiated using the corresponding constructor functions \texttt{newInt()}, \texttt{newBool()} \emph{etc}.\@ introduced above.
%
Operations on fundamental data types preserve value semantics by always allocating new memory for the result \texttt{TVP} instance and returning the corresponding pointer.
%
\paragraph{Structured types.}
Like basic types, aggregate types such as sets and maps are treated in exactly the same way.
%
The support library provides both the data type infrastructure as well as the operations on aggregate types such that translation is rendered straightforward.
%
For example, the expression
%
%
%
\begin{lstlisting}[language=VDM++,frame=tlbr]
a : set of int := {1} union {2}
\end{lstlisting}
%
%
%
translates directly to 
%
%
%
\begin{lstlisting}[language=C,frame=tlbr]
TVP a = vdmSetUnion(newSetVar(1, newInt(1)), newSetVar(1, newInt(2)));
\end{lstlisting}
%
%
%
where \texttt{newSetVar()} is one of the several special-purpose internal constructors.
%
The translation strategy is similar for sequences and maps.
%
Value semantics for these immutable data types is maintained in the same way as for the basic data types.
%
%
%
\paragraph{Quote types}
Quote types such as that shown in Listing \ref{lst:quoteex} are treated at the individual element level.
%
Each element is assigned a unique number via a \texttt{\#define} directive, as shown in Listing \ref{lst:quoteex}.
%
%
%
\begin{lstlisting}[language=VDM++,caption={Quote type example.},label={lst:quoteex},frame=tlbr]{lst:quoteex}
class QuoteExample

types

	public QuoteType = <Val1> | <Val2> | <Val3>

end QuoteExample
\end{lstlisting}
%
%
%
\begin{lstlisting}[language=VDM++,caption={Quote type example translation.},label={lst:quoteimpl},frame=tlbr]{lst:quoteimpl}
...

#ifndef QUOTE_VAL1
#define QUOTE_VAL1 2658640
#endif /* QUOTE_VAL1 */

#ifndef QUOTE_VAL2
#define QUOTE_VAL2 2658641
#endif /* QUOTE_VAL2 */

...
\end{lstlisting}
%
%
%
\paragraph{Union types.}
The decision to keep run-time type information for every variable of type \texttt{TVP} obviates the need for a translation strategy for union types.
%
%
%
\subsubsection{Translating Features of VDM++}
\paragraph{Classes}
%
Earlier we introduced the mechnism of C structures used to represent classes.
%
Translation of a model class is therefore straightforward, with each class receiving its own specific \texttt{struct}.
%
As illustrated in Listings \ref{lst:vdmexheader} and \ref{lst:vdmeximpl} above, each class receives its own pair of C header and implementation files.
%
Most importantly, the header file contains the definition of the corresponding class \texttt{struct} and the declarations of the interface functions for this struct.
%
These include the top-level constructor and initialization and cleanup functions for class \texttt{values} and \texttt{static} field declarations.
%
The implementation (\texttt{.c}) file contains the constructor mechanism, the definition of the virtual function table of the class and the implementations of the class's functions and operations.
%
The virtual function table is constructed in accordance with the inheritance hierarchy in which the class belongs (this is discussed below).
%
Class \texttt{values} definitions and static fields are implemented as global variables.
%
Their definitions are also inserted in the implementation file, along with initializer and cleanup functions to be called, respectively, when the implementation starts and terminates.

\paragraph{Inheritance}
%
The effect of inheritance is to augment the definition of the inheriting class with the features of the parent class, \emph{modulo} overriding.
%
In our \texttt{struct}-based implementation of classes and objects, the traits of the base class are copied into the \texttt{struct} corresponding to the inheriting class.
%
Therefore, the \texttt{struct} of the inheriting class duplicates the fields and virtual function table of the base class.

Consider the translation of the following model with inheritance.
%
Despite its cumbersome length, we provide the listing of the complete translation so that the reader may also gain familiarity (at his/her own pace) with all the elements of the generated code.
%
%
%
\begin{lstlisting}[language=VDM++,frame=tlbr]
class A
instance variables
public field_A : int := 0;

operations
public opA : int ==> int
opA(i) == return i;
end A


class B is subclass of A
operations
public opB : () ==> ()
opB() == skip;
end B


class C
instance variables
b : B := new B();

operations
public op : () ==> int
op() == return b.opA(b.field_A);
end C
\end{lstlisting}
%
%
%
The six files \texttt{A.h}, \texttt{A.c}, \texttt{B.h}, \texttt{B.c}, \texttt{C.h} and \texttt{C.c} reproduced below make up the complete translation.
%
%
%
\begin{lstlisting}[language=C,frame=tlbr,caption="File A.h."]
// The template for class header
#ifndef CLASSES_A_H_
#define CLASSES_A_H_

#define VDM_CG

#include "Vdm.h"

//include types used in the class
#include "A.h"


extern TVP numFields_1;

#define CLASS_ID_A_ID 0
#define ACLASS struct A*

#define CLASS_A__Z3opAEI 0

struct A
{
	VDM_CLASS_BASE_DEFINITIONS(A);

	VDM_CLASS_FIELD_DEFINITION(A,field_A);
	VDM_CLASS_FIELD_DEFINITION(A,numFields);
};

TVP _Z1AEV(ACLASS this_);

void A_const_init();
void A_const_shutdown();
void A_static_init();
void A_static_shutdown();

void A_free_fields(ACLASS);
ACLASS A_Constructor(ACLASS);

#endif /* CLASSES_A_H_ */

\end{lstlisting}
%
%
%
\begin{lstlisting}[language=C,frame=tlbr,caption="File A.c."]
#include "A.h"
#include <stdio.h>
#include <string.h>


void A_free_fields(struct A *this)
{
	vdmFree(this->m_A_field_A);
}

static void A_free(struct A *this)
{
	--this->_A_refs;
	if (this->_A_refs < 1)
	{
		A_free_fields(this);
		free(this);
	}
}

static  TVP _Z17fieldInitializer2EV(){
	TVP ret_1 = vdmClone(newInt(0));

	return ret_1;
}

static  TVP _Z17fieldInitializer1EV(){
	TVP ret_2 = vdmClone(newInt(1));

	return ret_2;
}

static  TVP _Z3opAEI(ACLASS this, TVP i){
	TVP ret_3 = vdmClone(i);

	return ret_3;
}

void A_const_init(){
	numFields_1 = _Z17fieldInitializer1EV();

	return ;
}

void A_const_shutdown(){
	vdmFree(numFields_1);

	return ;
}

void A_static_init(){

	return ;
}

void A_static_shutdown(){

	return ;
}

static  struct VTable VTableArrayForA  [] ={

		{0,0,((VirtualFunctionPointer) _Z3opAEI),},
}  ;

ACLASS A_Constructor(ACLASS this_ptr)
{
	if(this_ptr==NULL)
	{
		this_ptr = (ACLASS) malloc(sizeof(struct A));
	}

	if(this_ptr!=NULL)
	{
		this_ptr->_A_id = CLASS_ID_A_ID;
		this_ptr->_A_refs = 0;
		this_ptr->_A_pVTable=VTableArrayForA;

		this_ptr->m_A_field_A= _Z17fieldInitializer2EV();
	}

	return this_ptr;
}

static TVP new(){
	ACLASS ptr=A_Constructor(NULL);

	return newTypeValue(VDM_CLASS, (TypedValueType)
			{	.ptr=newClassValue(ptr->_A_id, &ptr->_A_refs,\
				(freeVdmClassFunction)&A_free, ptr)});
}

TVP _Z1AEV(ACLASS this){
	TVP __buf = NULL;

	if ( this == NULL )
	{
		__buf = new();

		this = TO_CLASS_PTR(__buf, A);
	}

	return __buf;
}

TVP numFields_1 =  NULL ;

\end{lstlisting}
%
%
%
\begin{lstlisting}[language=C,frame=tlbr,caption="File B.h."]
#ifndef CLASSES_B_H_
#define CLASSES_B_H_

#define VDM_CG

#include "Vdm.h"
#include "A.h"
#include "B.h"

#define CLASS_ID_B_ID 1

#define BCLASS struct B*

#define CLASS_B__Z3opBEV 0

struct B
{
	VDM_CLASS_BASE_DEFINITIONS(A);

	VDM_CLASS_FIELD_DEFINITION(A,field_A);
	VDM_CLASS_FIELD_DEFINITION(A,numFields);

	VDM_CLASS_BASE_DEFINITIONS(B);

	VDM_CLASS_FIELD_DEFINITION(B,numFields);
};

TVP _Z1BEV(BCLASS this_);

void B_const_init();
void B_const_shutdown();
void B_static_init();
void B_static_shutdown();

void B_free_fields(BCLASS);
BCLASS B_Constructor(BCLASS);

#endif /* CLASSES_B_H_ */

\end{lstlisting}
%
%
%
\begin{lstlisting}[language=C,frame=tlbr,caption="File B.c."]
#include "B.h"
#include <stdio.h>
#include <string.h>

void B_free_fields(struct B *this)
{
}

static void B_free(struct B *this)
{
	--this->_B_refs;
	if (this->_B_refs < 1)
	{
		B_free_fields(this);
		free(this);
	}
}

static  void _Z3opBEV(BCLASS this){
	{
		//Skip
	};
}

void B_const_init(){
	return ;
}

void B_const_shutdown(){
	return ;
}

void B_static_init(){
	return ;
}

void B_static_shutdown(){
	return ;
}

static  struct VTable VTableArrayForB  [] ={

		{0,0,((VirtualFunctionPointer) _Z3opBEV),},
}  ;

BCLASS B_Constructor(BCLASS this_ptr)
{
	if(this_ptr==NULL)
	{
		this_ptr = (BCLASS) malloc(sizeof(struct B));
	}

	if(this_ptr!=NULL)
	{
		A_Constructor((ACLASS)CLASS_CAST(this_ptr,B,A));

		this_ptr->_B_id = CLASS_ID_B_ID;
		this_ptr->_B_refs = 0;
		this_ptr->_B_pVTable=VTableArrayForB;
	}

	return this_ptr;
}

static TVP new(){
	BCLASS ptr=B_Constructor(NULL);

	return newTypeValue(VDM_CLASS, (TypedValueType)
			{	.ptr=newClassValue(ptr->_B_id, &ptr->_B_refs,\
					(freeVdmClassFunction)&B_free, ptr)});
}

TVP _Z1BEV(BCLASS this){
	TVP __buf = NULL;

	if ( this == NULL )
	{
		__buf = new();
		this = TO_CLASS_PTR(__buf, B);
	}

	_Z1AEV(((ACLASS) CLASS_CAST(this, B, A)));

	return __buf;
}

\end{lstlisting}
%
%
%
\begin{lstlisting}[language=C,frame=tlbr,caption="File C.h."]
#ifndef CLASSES_C_H_
#define CLASSES_C_H_

#define VDM_CG

#include "Vdm.h"
#include "B.h"
#include "C.h"

extern TVP numFields_2;

#define CLASS_ID_C_ID 2

#define CCLASS struct C*

#define CLASS_C__Z2opEV 0

struct C
{
	VDM_CLASS_BASE_DEFINITIONS(C);

	VDM_CLASS_FIELD_DEFINITION(C,b);
	VDM_CLASS_FIELD_DEFINITION(C,numFields);
};

TVP _Z1CEV(CCLASS this_);

void C_const_init();
void C_const_shutdown();
void C_static_init();
void C_static_shutdown();

void C_free_fields(CCLASS);
CCLASS C_Constructor(CCLASS);

#endif /* CLASSES_C_H_ */

\end{lstlisting}
%
%
%
\begin{lstlisting}[language=C,frame=tlbr,caption="File C.c."]
#include "C.h"
#include <stdio.h>
#include <string.h>

void C_free_fields(struct C *this)
{
	vdmFree(this->m_C_b);
}

static void C_free(struct C *this)
{
	--this->_C_refs;
	if (this->_C_refs < 1)
	{
		C_free_fields(this);
		free(this);
	}
}

static  TVP _Z17fieldInitializer4EV(){
	TVP ret_4 = vdmClone(_Z1BEV(NULL));

	return ret_4;
}

static  TVP _Z17fieldInitializer3EV(){
	TVP ret_5 = vdmClone(newInt(1));

	return ret_5;
}

static  TVP _Z2opEV(CCLASS this){
	TVP embeding_1 = \
		GET_FIELD(A, A, GET_FIELD_PTR(C, C, this, b), field_A);

	TVP ret_6 = vdmClone(CALL_FUNC(B, A, GET_FIELD_PTR(C, C, this, b), \
			CLASS_A__Z3opAEI, embeding_1));

	return ret_6;
}

void C_const_init(){
	numFields_2 = _Z17fieldInitializer3EV();

	return ;
}

void C_const_shutdown(){
	vdmFree(numFields_2);

	return ;
}

void C_static_init(){
	return ;
}

void C_static_shutdown(){
	return ;
}

static  struct VTable VTableArrayForC  [] ={

		{0,0,((VirtualFunctionPointer) _Z2opEV),},
}  ;

CCLASS C_Constructor(CCLASS this_ptr)
{
	if(this_ptr==NULL)
	{
		this_ptr = (CCLASS) malloc(sizeof(struct C));
	}

	if(this_ptr!=NULL)
	{
		this_ptr->_C_id = CLASS_ID_C_ID;
		this_ptr->_C_refs = 0;
		this_ptr->_C_pVTable=VTableArrayForC;

		this_ptr->m_C_b= _Z17fieldInitializer4EV();
	}

	return this_ptr;
}

static TVP new(){
	CCLASS ptr=C_Constructor(NULL);

	return newTypeValue(VDM_CLASS, (TypedValueType)
			{	.ptr=newClassValue(ptr->_C_id, &ptr->_C_refs,\
					(freeVdmClassFunction)&C_free, ptr)});
}

TVP _Z1CEV(CCLASS this){
	TVP __buf = NULL;

	if ( this == NULL )
	{
		__buf = new();

		this = TO_CLASS_PTR(__buf, C);
	}

	return __buf;
}

TVP numFields_2 =  NULL ;
\end{lstlisting}
%
%
%
The duplication of the elements of \texttt{A} can be seen in the definition of \texttt{struct B} in \texttt{B.h}.
%
The listing for \texttt{C.c} illustrates the mechanism by which a call to an inherited operation on an instance of \texttt{B} is achieved.
%
The macro \texttt{CALL\_FUNC} is the primary function and method call mechanism.
%
It uses information about the type of the object on which the operation is invoked, as well as the class in which the operation is actually defined, to calculate a function pointer offset in the correct (duplicated) virtual function table of the instance of \texttt{B}.
%
The class in which the operation is originally defined (\texttt{A} in this case) is calculated by scanning the chain of superclasses of \texttt{B} and choosing the nearest definition.
%
This method satisfies semantics of calls of inherited operations.
%
\paragraph{Polymorphism}
Overture limits polymorphism, the overloading of operations and functions.
%
Overloaded operations and functions can only be distinguished by Overture's type system only if they differ in their parameter types.
%
Operations differing only in return type can not be distinguished, rendering the following example definition illegal.
%
%
%
\begin{lstlisting}[language=VDM++,frame=tlbr]
class Overloading

operations

public op : bool ==> ()
op(a) == skip;

public op : bool ==> bool
op(a) == return true;

end Overloading
\end{lstlisting}
%
%
%
Polymorphism is implemented by way of a name mangling scheme, whereby the name generated for any operation or function is augmented with tags representing its parameter types.
%
For instance, the name of the following operation
%
%
%
\begin{lstlisting}[language=VDM++,frame=tlbr]
public theOperation : int * bool * char ==> real
\end{lstlisting}
%
%
%
is generated as \texttt{\_Z12theOperationEIBC}.
%
The mangled name can be decomposed as follows:
%
%
%
\begin{itemize}
\item  \texttt{\_Z}:  prepended to all mangled names.
\item  \texttt{12}:  number of characters in the original name.
\item  \texttt{theOperation}:  the original name.
\item  \texttt{E}:  separator between name and parameter type tags.
\item  \texttt{I}:  \texttt{int} parameter.
\item  \texttt{B}:  \texttt{bool} parameter.
\item  \texttt{C}:  \texttt{char} parameter.
\end{itemize}

\paragraph{Function and operation overriding}
%
In single inheritance scenarios, operation/funciton overriding is achieved in a simple way by choosing the overriding implementation closest in the inheritance chain to the class to which the object on which the operation is invoked belongs.
%
This is in accordance with the corresponding semantics.
%
In multiple inheritance scenarios, Overture does not allow ambiguity leading to a choice of implementation.
%
For instance, the following model is illegal in Overture:
%
%
%
\begin{lstlisting}[language=VDM++,frame=tlbr]
class A
operations
public op : () ==> bool
op() ==
	return true
end A

class B
operations
public op : () ==> bool
op() ==
	return false
end B

class C is subclass of B, A
end C
\end{lstlisting}
%
This forces the model developer to eliminate all such ambiguity, reducing the scenario that that of single inheritance.
