%
%  Simplified Overture Community Process
%
%  Created by Miguel Ferreira on 2009-02-06.
%  Copyright (c) 2009 __MyCompanyName__. All rights reserved.
%
\documentclass[]{article}
\usepackage{a4}
% package that enables the use of colors
\usepackage{color}

\newcommand{\blue}[1]{{\color{blue} #1}}
\newcommand{\reminder}[1]{\fbox{\blue{#1}}}

\title{The Overture Community Process}
\author{Miguel Ferreira and John Fitzgerald}

\date{Release 1 \\
      Adopted April 2009}

\begin{document}

\maketitle

\section*{Preface}
The OML is a formal language for the modelling of computing
systems. The OCP governs the management of the OML definition
documents and hence has an effect on the conformance of the emerging
tools that support modelling and analysis in OML. In particular, the
OCP balances the need to encourage experimentation and sound extension
of the language against the imperative for rigour in the definition of
its syntax and semantics. As a result, the process involves two review
stages. During the first stage, the modification request is managed by
a trusted closed group. The second phase of review is open to the
whole community.

\subsubsection*{Acronyms}
\begin{tabular}{ll}
LB  & Language Board \\
OCP & Overture Community Process \\
OML & Overture Modelling Language \\
RM  & Request for Modification
\end{tabular}

\section{Target} % (fold)
\label{sec:target}

The OCP targets the management of the definition documents for the OML
and hence for the emerging set of tools that support modelling and
analysis in OML. The process is therefore relevant to all those who
are working on the language itself and the tools to support its use.

\section{Membership} % (fold)
\label{sec:members_organization}
\begin{enumerate}
\item The Overture Community shall consist of persons called
  \emph{members}. The members are those persons who subscribe to the
  Overture Core mailing list.

\item There shall be a committee of not more than \textbf{five}
  Overture Community members called the \emph{Language Board}.

\item The LB shall step down on a specified date \textbf{annually}.

\item If more than five candidates present themselves for LB
  membership, an election shall take place~(Section~\ref{sub:voting}). 

\end{enumerate}

\section{Operation of the Language Board}
\label{sec:exec} 

\begin{enumerate}
\item The LB shall appoint the following from among its members:
\begin{description}
		\item[A Convener] who organises and chairs LB
                  meetings. The Convener may not be the same person as
                  the Secretary.
		\item[A Secretary] who manages the business, recording
                  LB decisions and reporting them to the
                  community.
	\end{description}
\item The LB is responsible for the timely consideration of all
  RMs. 
\end{enumerate}

\section{Workflow of a Request for Modification} % (fold)
\label{sec:workflow_of_a_request_for_modification}

In the following workflow, members have rights and responsibilities as follows: 
\begin{enumerate}
\item Any Overture Community member may submit an RM at any time. The member
submitting an RM is termed the RM's \emph{Originator}.

\item The LB must process all RMs equally and in timely manner. 

\item The Originator may withdraw a submitted RM at any time prior to termination of
the process by informing the Convener and Secretary. 

\item The LB may change the status of an RM to \textbf{rejected} at any time
after submission and terminate the process, giving its reasons to the
Originator.
\end{enumerate}

\subsection{Definition of Request for Modification} % (fold)
\label{sub:definition_of_request_for_modification}

The LB shall specify the required content of RMs. As a minimum, the
following shall be required:
\begin{enumerate}
	\item Identification of the Originator.
	
	\item Target of the request: defining the affected components
          of the language definition.
	
	\item Motivation for the request.
	
	\item Description of the request, including:
	\begin{enumerate}
		\item description of the modification;
		
		\item benefits from the modification;
		
		\item possible side effects.
	\end{enumerate}
	
%	\item Source code and technical documentation where applicable.
	
	\item If appropriate, a test suite for validation of the
          modification in the executable models. 
\end{enumerate}
% subsubsection definition_of_request_for_modification (end)

\subsection{Submission} % (fold)
\label{sub:submitting_a_request_for_modification}

\begin{enumerate}
\item The submission of an RM shall be done through a form on the
  Overture web site (possibly using an issue tracking system), that
  causes a notification email to be sent to the members of the LB only. 

\item Submissions requiring privacy or limited circulation during
  initial consideration shall be discussed between the Originator and
  the Convener prior to submission, to ensure that these requirements
  are, so far as possible, respected.
\end{enumerate}

\subsection{Initial Consideration}
\label{sec:initial}
\begin{enumerate}
\item After submission the RM is evaluated by the LB. The LB may
  request expert opinions from named members, subject to the agreement
  of the Originator.
\item The LB may return one of three verdicts: 
\begin{enumerate}
\item The RM may be \textbf{rejected}. The Secretary shall communicate the
  verdict to the Originator with explanation. The process terminates.
\item The RM may be \textbf{approved unmodified}. The Secretary
  shall communicate the verdict to the Originator, giving a date when the
  RM will be moved to the Discussion phase.
\item The RM may be \textbf{approved subject to revision}. The
  Secretary shall communicate the verdict to the Originator. The Originator
  may revise the RM in accordance with the changes suggested by the LB
  and pass it back to the Secretary for final approval. The Convener
  may approve such an RM independently or by reference to the full LB.
\end{enumerate}
\end{enumerate}
% subsubsection submitting_a_request_for_modification (end)

\subsection{Discussion} % (fold)
\label{sub:discussion_of_a_request_for_modification}
\begin{enumerate} 
\item An RM that enters Discussion phase shall be made public through the
  Overture web site and a discussion mailing list, according to the
  component(s) that it affects. The mailing lists for discussion can
  be subscribed by any member.
\item The LB Secretary shall start the discussion of an RM by
  publishing the following to all Overture Community members: 
\begin{enumerate}
\item the RM as approved;
\item reasons for the approval of the RM;
\item any technical issues identified during the initial consideration
  phase; and
\item a date for the conclusion of discussion.
\end{enumerate}
\item The discussion shall then be carried out in a constructive
  manner by the Overture Community members, terminating at the
  specified date.
\end{enumerate}
% subsubsection discussion_of_a_request_for_modification (end)

\subsection{Deliberation} % (fold)
\label{sub:deliberation_of_a_request_for_modifiation}
The LB shall consider the RM, taking into account the content of the
Discussion phase. The LB may return one of three verdicts:
\begin{enumerate}
\item The RM may be \textbf{rejected}. The process terminates. 
\item The RM may be \textbf{accepted unmodified}. The RM passes to the
  Execution phase.
\item The RM may be \textbf{accepted subject to revision}. The
  Secretary shall communicate the verdict to the Originator, giving
  required revisions. The Originator may revise the RM in
  accordance with the changes suggested by the LB and pass it back to
  the Secretary for final acceptance. The Convener may accept such an
  RM independently or by reference to the full LB. After this point,
  the RM is treated as \textbf{accepted unmodified}.
\end{enumerate}

\subsection{Execution} % (fold)
\label{sub:execution}

The LB shall appoint a team of members (including the Originator) to
carry out the RM and validate its successful completion. The team
reports to the LB.  The team may report \textbf{completion}, including
the performance of validation tests as required in the RM. In this
case, a new version (tag in svn terms) of the Overture repository is
released. The process terminates.  The team may report that it is
\textbf{not able to complete the RM successfully}. In this case, the
RM shall be referred back to the LB for a decision on how to proceed.

\section{Voting} % (fold)
\label{sub:voting}
\begin{description}
\item[\textbf{Election of the LB}] shall be by Single Transferable Vote. All
members shall have one vote. The election process shall be managed by
a member not standing for election who is nominated and approved by
simple majority voting in public open to all Overture Community members. 

\item[\textbf{Decisions within the LB}] shall be by simple majority
voting. 

\item[\textbf{Changes to this process}] shall be agreed by simple majority
vote open to all Overture Community members. 
\end{description}
% subsection voting (end)
% workflow_of_a_request_for_modification (end)

\end{document}
