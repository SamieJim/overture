To conclude, the issues laid out in the introduction and background chapters are eliminated by this tool, the aims and objectives of this dissertation achieved. As explained in the previous section; the development of the tool was difficult due the vast complexity of its environment. Considering this, and considering that the tool works correctly, this project should be considered a success, this conclusion will attempt justify this statement. 

\section{Measuring The Success Of This Project}
From the aims and objectives defined in the Introduction chapter all were achieved. The aim of this dissertation was to extend the current VDM to Isabelle translation tool so that it can translate more components of the POLAR model, this has been achieved and arguably exceeded, as more constructs than are present in the VDM-SL POLAR model can be translated. The objectives of this project were:
\begin{itemize}
	\item Understand the Java visitor design pattern.
	\item Understand the existing tool architecture.
	\item Create Velocity templates and Java visitors for more VDM constructs.
	\item Apply the translator to the POLAR model.
\end{itemize}

All of which were achieved. I fully understand the visitor design pattern and the existing tool architecture. Though I have created two Velocity templates and modified more, it was not necessary, and in fact would have been irresponsible, to modify or create too much Velocity as a good template is just that, a template, it should be abstract, short, and minimal. Large amounts of Java was written in existing visitors and a few more created, both for transformation and utility. Application to the POLAR model was a success, it is translated correctly with the exception of some minor foibles already covered, caused by a few kinks in the tool.

\section{Future Development}
This tool lays the foundations for more sophisticated prototypes to be developed on top of it. Flexible methods and translation of basic VDM constructs have made the tool extensible, but in some areas, like the way that symbols are reorganised to their correct positions in set and sequence initialisation, the tool uses a "hacky" method to iterate through where they should have been in their correct place to begin with.\footnote{This quirk of the tool is a side effect of reverse recursion to generate symbols.} Though they present no practical obstacle and the tool works well for the tests that have been run, given more time, these inelegant methods could be transformed or removed entirely after some thought so that they do not present potential problems down the line for larger models. For actual additional functionality however, it would be highly beneficial if this tool could also generate lemmas, the units of auxiliary information used by Isabelle to prove a model. This would be done in an approximately similar way to the way that it has been done for invariant generations during the development of this tool, although they would have to be generated \emph{after} the generation of the translated model. The reason for this, is that the lemmas are based on other constructs in the Isabelle file as well as theories present in the "Main" Isabelle theory directory mentioned previously. 

Should lemmas be generated successfully, the tool would become more than a translation engine and code generator. The tool would then, with the assistance of the Isabelle automated theorem assistant, become an automated proof tool that could have vastly beneficial implementations in proving the specification of safety critical systems.  


\section{Impact of This Project}
The introduction explains that before the development of this tool, manually translate an Isabelle model. This is cumbersome for two reasons, the first is that the modeller is human and makes mistakes, despite the high intelligence true of those working in the field of mathematical formal specification, human error is ever present. Small errors made in translation could result in faults with proof, e.g. proving the wrong thing. This means that the soundness of proof of an Isabelle proven specification could come into question. The second reason is the massive additional time required to translate a model into Isabelle. Each construct must be manually typed and considered, this means that it simply is not feasible to translate large Isabelle models, like the one for POLAR into Isabelle. The important thing to note is that the man hours required to translate a model could not only be spent proving it, which is the difficult part, but if we are to use Isabelle to prove all safety critical systems, as we should to ensure the upmost safety assurance, then a potentially life saving/changing technology could be delayed while such an automatable step is performed. This tool remedies both, for the first, automation means that we only have to prove that a translation is sound once, when we create the translation recipe that the tool follows, then any subsequent translations done by the tool are always likely to be correct, practically eliminating the concern of modeller error. The second encumbrance of time is removed entirely thanks to the speed of the computer, automated translation could speed up the development of safety critical systems with the beneficial side effect of leaving the modeller more time to focus on proof.
