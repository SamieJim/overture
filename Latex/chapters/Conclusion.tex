To conclude, the issues laid out in the introduction and background chapters are eliminated by this tool, the aims and objectives of this dissertation achieved. As explained in the previous section; the development of the tool was difficult due the vast complexity of its environment. Considering this, and considering that the tool works correctly, the project should be considered a success, this conclusion will attempt justify this statement. 

\section{Measuring The Success Of This Project}
From the aims and objectives defined in the Introduction chapter, all were achieved. The aim of this dissertation was to extend the current VDM to Isabelle translation tool so that it could translate more components of the POLAR model, this went unchanged and has been achieved and exceeded as more models than just POLAR can be translated successfully, as demonstrated in \ref{atp}. The objectives of this project were augmented as development and research into the field progressed. The objectives before this project were:
\begin{itemize}
	\item Understand the Java visitor design pattern.
	\item Understand the existing tool architecture.
	\item Create Velocity templates and Java visitors for more VDM constructs.
	\item Apply the translator to the POLAR model.
\end{itemize}

These objectives were not enough to cover the work required for a project of such complexity, and in reflection I believe that I underestimated the task with these objectives. The objectives were changed to those seen in \ref{objectives}, it was no where near enough to understand solely the Java visitor design pattern. A thorough understanding of not only this pattern, but also its intricate implementation with cases and various adaptors, was required to create the functionality of the tool. Likewise, understanding the existing tool architecture did not come close to satisfying the depth of knowledge required to safely navigate and manipulate the AST - it is for this reason that it became an objective to understand the AST and the VDM modules which create and alter its structure. Though, as this paper has shown, I had created a number of Velocity templates and modified many more, it was not necessary, and in fact would have been irresponsible, to create too much Velocity because a good template is just that, a template, it should be abstract, short, and minimal. Working with the existing velocity at the start of the project, and making small modifications to it, improved the overall quality of the tool, and for this reason the third objective was removed. 

As was hopefully demonstrated in \ref{vp}, I fully understand the visitor design pattern and its implementation with adaptor classes in Overture/VDM, \ref{wd} should have provided an argument that this knowledge enabled me to create transformations to the AST. \ref{ead} shows how I successfully learnt a large amount about the AST's architecture and the VDM modules which create and manipulate it - again, subsequent sections in \ref{wd} should demonstrate a well learned, clear ability to work with the AST correctly. Application to the POLAR model was a success and the tool even went further to translate Alarm.vdmsl, FSM3.vdmsl and by extension could translate countless more models. Translation is successful and efficient with the exception of some small foibles already covered, caused by a few kinks in the tool.

\section{Future Development}
This tool lays the foundations for more sophisticated prototypes to be developed on top of it. Flexible methods and translation of basic VDM constructs have made the tool extensible, but in some areas, like the way that symbols are reorganised to their correct positions in set and sequence initialisation, the tool uses a "hacky" method to iterate through where they should have been in their correct place to begin with.\footnote{This quirk of the tool is a side effect of reverse recursion to generate symbols.} Though they present no practical obstacle, and the tool works well for the tests that have been run, given more time, these inelegant methods could be transformed or removed entirely after some thought so that they do not present potential problems down the line for larger models. For actual additional functionality however, it would be highly beneficial if this tool could also generate lemmas, the units of auxiliary information used by Isabelle to prove a model. This would be done in an similar way to the way that it has been done for invariant generations during the development of this tool, although they would have to be generated \emph{after} the generation of the translated model. The reason for this, is that the lemmas are based on other constructs in the Isabelle file as well as theories present in the "Main" Isabelle theory directory mentioned previously. 

Should lemmas be generated successfully, the tool would become more than a translation engine and code generator. The tool would then, with the assistance of the Isabelle automated theorem assistant, become an automated proof tool that could have vastly beneficial implementations in proving the specification of safety critical systems.  


\section{Impact of This Project}
The introduction explains that before the development of this tool, the modeller was required to manually translate an VDM-SL model into Isabelle. This is cumbersome for two reasons, the first is that the modeller is human and makes mistakes, despite the high intelligence true of those working in the field of mathematical formal specification, human error is ever present. Small errors made in translation could result in faults with proof, e.g. proving the wrong thing. This means that the soundness of proof of an Isabelle proven specification could come into question. The second reason is the massive additional time required to translate a model into Isabelle. Each construct must be manually typed and considered, this means that it simply is not feasible to translate large Isabelle models, like the one for POLAR into Isabelle. The important thing to note is that the man hours required to translate a model could not only be spent proving it, which is the difficult part, but if we are to use Isabelle to prove all safety critical systems, as we should to ensure the utmost safety assurance, then a potentially life saving/changing technology could be delayed while such an automatable step is performed. This tool remedies both, for the first, automation means that we only have to prove that a translation is sound once, when we create the translation recipe that the tool follows, then any subsequent translations done by the tool are always likely to be correct, practically eliminating the concern of modeller error. The second encumbrance of time is removed entirely thanks to the speed of the computer, automated translation could speed up the development of safety critical systems with the beneficial side effect of leaving the modeller more time to focus on proof.
