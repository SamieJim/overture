% !TEX TS-program = pdflatex
% !TEX encoding = UTF-8 Unicode

% This is a simple template for a LaTeX document using the "article" class.
% See "book", "report", "letter" for other types of document.

\documentclass[11pt]{article} % use larger type; default would be 10pt

\usepackage[utf8]{inputenc} % set input encoding (not needed with XeLaTeX)


%%% PAGE DIMENSIONS
\usepackage{geometry} % to change the page dimensions
\geometry{a4paper} % or letterpaper (US) or a5paper or....


\usepackage{graphicx} % support the \includegraphics command and options


%%% PACKAGES
\usepackage{booktabs} % for much better looking tables
\usepackage{array} % for better arrays (eg matrices) in maths
\usepackage{paralist} % very flexible & customisable lists (eg. enumerate/itemize, etc.)
\usepackage{verbatim} % adds environment for commenting out blocks of text & for better verbatim
\usepackage{subfig} % make it possible to include more than one captioned figure/table in a single float
% These packages are all incorporated in the memoir class to one degree or another...

%%% HEADERS & FOOTERS
\usepackage{fancyhdr} % This should be set AFTER setting up the page geometry
\pagestyle{fancy} % options: empty , plain , fancy
\renewcommand{\headrulewidth}{0pt} % customise the layout...
\lhead{}\chead{}\rhead{}
\lfoot{}\cfoot{\thepage}\rfoot{}




\title{UML2 Translater for VDM}
\author{Kenneth Lausdahl}
%\date{} % Activate to display a given date or no date (if empty),
         % otherwise the current date is printed 

\begin{document}
\maketitle

\section{Type translation}

\subsection{Translation without use of associations}
\paragraph{Basic Types} Basic types are translated directly into UML primitive types:

\begin{tabular}{ l|l }
VDM Type name & UML Primitive type name\\\hline
  bool & bool \\
  char&char\\
token&token\\
\\
int&int\\
nat&nat\\
nat1&nat1\\
rat&rat\\
real&real\\
\\
\textless quote-name\textgreater&\textless quote-name\textgreater\\
\end{tabular}

\paragraph{Other} The other types available in VDM is translated as shown below, some of the VDM types like \textit{product types} does not have a fixed number of arguments and thus this translation will generate generic classes with the needed number of template parameters:

\begin{tabular}{ l|l }
VDM Type & UML Class name\\\hline
[int] & Optional\textless int\textgreater\\
set of int & Set\textless int\textgreater\\
seq of int & Seq\textless int\textgreater\\
 int\textbar int\textbar ... & Union\textless int,int,...\textgreater\\
 int*int*... & Product\textless int,int,...\textgreater\\
map int to int & Map\textless int,int\textgreater\\
inmap int to int & InMap\textless int,int\textgreater\\
() & Void\\
? & Any\\
\end{tabular}


\subsubsection{Associations}
An association will only be created if the types involved fulfils a number of constraints:
\begin{enumerate}
%\item Differs from \textit{vdm basic types}
%\item Not a function
\item Is a named invariant type
\item In the case of a map type both types used within the map must be: A set or seq of a named invariant type  or just a named invariant type
%\item Not a operation type
\item Can be an optional type if the inner type is a named invariant type
%\item not a parameter type
%\item not a product type
%\item not a quote type
\item Can be a set type if the inner type is a named invariant type
\item Can be a seq type if the inner type is a named invariant type
\item None of the following types are allowed:
	\begin{enumerate}
		\item Function
		\item All basic types
		\item Operation
		\item Parameter
		\item Product
		\item Quote
		\item Undefined
		\item Union
		\item Unknown
		\item Unresolved
		\item Void
		\item VoidReturn
	\end{enumerate}
\end{enumerate}

Future more associations are only created if the types are named invariants unless the custom option for preferred associations is enabled.
\section{Not supported yet}
\begin{itemize}
	\item Polymorphic functions
\end{itemize}


\end{document}
